\chapter{Tutorial\label{tutorial}}

\begin{flushright}
  \emph{So how can I make this work?}
\end{flushright}

\emph{This is a quick guide to getting started with mod_python 
  programming once you have it installed. This is \textbf{not} an
  installation manual!}

\emph{It is also highly recommended to read (at least the top part of)
  Section \ref{pythonapi}, \citetitle[pythonapi.html]{Python API} after
  completing this tutorial.}

\section{A Quick Start with the Publisher Handler\label{tut-pub}}

This section provides a quick overview of the Publisher handler for
those who would like to get started without getting into too much
detail. A more thorough explanation of how mod_python handlers work
and what a handler actually is follows on in the later sections of the
tutorial.

The \code{publisher} handler is provided as one of the standard
mod_python handlers. To get the publisher handler working, you will
need the following lines in your config:

\begin{verbatim}
  AddHandler mod_python .py
  PythonHandler mod_python.publisher
  PythonDebug On
\end{verbatim}

The following example will demonstrate a simple feedback form. The
form will ask for the name, e-mail address and a comment and construct
an e-mail to the webmaster using the information submitted by the
user. This simple application consists of two files:
\filenq{form.html} - the form to collect the data, and
\filenq{form.py} - the target of the form's action.

Here is the html for the form:

\begin{verbatim}
  <html>
      Please provide feedback below:
  <p>                           
  <form action="form.py/email" method="POST">

      Name:    <input type="text" name="name"><br>
      Email:   <input type="text" name="email"><br>
      Comment: <textarea name="comment" rows=4 cols=20></textarea><br>
      <input type="submit">

  </form>
  </html>  
\end{verbatim}

Note the \code{action} element of the \code{<form>} tag points to
\code{form.py/email}. We are going to create a file called
\filenq{form.py}, like this:

\begin{verbatim}
import smtplib

WEBMASTER = "webmaster"   # webmaster e-mail
SMTP_SERVER = "localhost" # your SMTP server

def email(req, name, email, comment):

    # make sure the user provided all the parameters
    if not (name and email and comment):
        return "A required parameter is missing, \
	       please go back and correct the error"

    # create the message text
    msg = """\
From: %s                                                                                                                                           
Subject: feedback
To: %s

I have the following comment:

%s

Thank You,

%s

""" % (email, WEBMASTER, comment, name)

    # send it out
    conn = smtplib.SMTP(SMTP_SERVER)
    conn.sendmail(email, [WEBMASTER], msg)
    conn.quit()

    # provide feedback to the user
    s = """\
<html>

Dear %s,<br>                                                                                                                                       
Thank You for your kind comments, we
will get back to you shortly.

</html>""" % name

    return s
\end{verbatim}

When the user clicks the Submit button, the publisher handler will
load the \function{email} function in the \module{form} module,
passing it the form fields as keyword arguments. It will also pass the
request object as \code{req}.

Note that you do not have to have \code{req} as one of the arguments
if you do not need it. The publisher handler is smart enough to pass
your function only those arguments that it will accept.

The data is sent back to the browser via the return value of the
function.

Even though the Publisher handler simplifies mod_python programming a
great deal, all the power of mod_python is still available to this
program, since it has access to the request object. You can do all the
same things you can do with a ``native'' mod_python handler, e.g. set
custom headers via \code{req.headers_out}, return errors by raising
\exception{apache.SERVER_ERROR} exceptions, write or read directly to
and from the client via \method{req.write()} and \method{req.read()},
etc.

Read Section \ref{hand-pub} \citetitle[hand-pub.html]{Publisher Handler}
for more information on the publisher handler. 

\section{Quick Overview of how Apache Handles Requests\label{tut-overview}}

If you would like delve in deeper into the functionality of
mod_python, you need to understand what a handler is.  

Apache processes requests in \dfn{phases}. For example, the first
phase may be to authenticate the user, the next phase to verify
whether that user is allowed to see a particular file, then (next
phase) read the file and send it to the client. A typical static file
request involves three phases: (1) translate the requested URI to a
file location (2) read the file and send it to the client, then (3)
log the request. Exactly which phases are processed and how varies
greatly and depends on the configuration.

A \dfn{handler} is a function that processes one phase. There may be
more than one handler available to process a particular phase, in
which case they are called by Apache in sequence. For each of the
phases, there is a default Apache handler (most of which by default
perform only very basic functions or do nothing), and then there are
additional handlers provided by Apache modules, such as mod_python.

Mod_python provides every possible handler to Apache. Mod_python
handlers by default do not perform any function, unless specifically
told so by a configuration directive. These directives begin with
\samp{Python} and end with \samp{Handler}
(e.g. \code{PythonAuthenHandler}) and associate a phase with a Python
function. So the main function of mod_python is to act as a dispatcher
between Apache handlers and Python functions written by a developer
like you.

The most commonly used handler is \code{PythonHandler}. It handles the
phase of the request during which the actual content is
provided. Because it has no name, it is sometimes referred to as as
\dfn{generic} handler. The default Apache action for this handler is
to read the file and send it to the client. Most applications you will
write will override this one handler. To see all the possible
handlers, refer to Section \ref{directives},
\citetitle[directives.html]{Apache Directives}.

\section{So what Exactly does Mod-python do?\label{tut-what-it-do}}

Let's pretend we have the following configuration: 
\begin{verbatim}
  <Directory /mywebdir>
      AddHandler mod_python .py
      PythonHandler myscript
      PythonDebug On
  </Directory>
\end{verbatim}

\strong{NB:} \filenq{/mywebdir} is an absolute physical path. 

And let's say that we have a python program (Windows users: substitute
forward slashes for backslashes) \file{/mywedir/myscript.py} that looks like
this:

\begin{verbatim}
from mod_python import apache

def handler(req):

    req.content_type = "text/plain"
    req.write("Hello World!")

    return apache.OK
\end{verbatim}    

Here is what's going to happen: The \code{AddHandler} directive tells
Apache that any request for any file ending with \file{.py} in the
\file{/mywebdir} directory or a subdirectory thereof needs to be
processed by mod_python. The \samp{PythonHandler myscript} directive
tells mod_python to process the generic handler using the
\code{myscript} script. The \samp{PythonDebug On} directive instructs
mod_python in case of an Python error to send error output to the
client (in addition to the logs), very useful during development.

When a request comes in, Apache starts stepping through its request
processing phases calling handlers in mod_python. The mod_python
handlers check whether a directive for that handler was specified in
the configuration. (Remember, it acts as a dispatcher.)  In our
example, no action will be taken by mod_python for all handlers except
for the generic handler. When we get to the generic handler,
mod_python will notice \samp{PythonHandler myscript} directive and do
the following:

\begin{enumerate}

\item
  If not already done, prepend the directory in which the
  \code{PythonHandler} directive was found to \code{sys.path}.

\item
  Attempt to import a module by name \code{myscript}. (Note that if
  \code{myscript} was in a subdirectory of the directory where
  \code{PythonHandler} was specified, then the import would not work
  because said subdirectory would not be in the \code{sys.path}. One
  way around this is to use package notation, e.g. \samp{PythonHandler
  subdir.myscript}.)

\item 
  Look for a function called \code{handler} in \code{myscript}.

\item
  Call the function, passing it a request object. (More on what a
  request object is later)

\item
  At this point we're inside the script: 

  \begin{itemize}

  \item
    \begin{verbatim}
from mod_python import apache
    \end{verbatim}

    This imports the apache module which provides us the interface to
    Apache. With a few rare exceptions, every mod_python program will have
    this line.

  \item
    \begin{verbatim}
def handler(req):
    \end{verbatim}

    \index{handler} This is our \dfn{handler} function declaration. It
    is called \samp{handler} because mod_python takes the name of the
    directive, converts it to lower case and removes the word
    \samp{python}. Thus \samp{PythonHandler} becomes
    \samp{handler}. You could name it something else, and specify it
    explicitly in the directive using \samp{::}. For example, if the
    handler function was called \samp{spam}, then the directive would
    be \samp{PythonHandler myscript::spam}.

    Note that a handler must take one argument - the request
    object. The request object is an object that provides all of the
    information about this particular request - such as the IP of
    client, the headers, the URI, etc. The communication back to the
    client is also done via the request object, i.e. there is no
    ``response'' object.

  \item
    \begin{verbatim}
req.content_type = "text/plain"
    \end{verbatim}

    This sets the content type to \samp{text/plain}. The default is usually
    \samp{text/html}, but since our handler doesn't produce any html,
    \samp{text/plain} is more appropriate.

  \item
    \begin{verbatim}
req.write("Hello World!")
    \end{verbatim}

    This writes the \samp{Hello World!} string to the client. (Did I really
    have to explain this one?)

  \item
    \begin{verbatim}
return apache.OK
    \end{verbatim}

    This tells Apache that everything went OK and that the request has
    been processed. If things did not go OK, that line could be return
    \constant{apache.HTTP_INTERNAL_SERVER_ERROR} or return
    \constant{apache.HTTP_FORBIDDEN}. When things do not go OK, Apache
    will log the error and generate an error message for the client.
  \end{itemize}
\end{enumerate}

\strong{Some food for thought:} If you were paying attention, you
noticed that the text above didn't specify that in order for the
handler code to be executed, the URL needs to refer to
\filenq{myscript.py}. The only requirement was that it refers to a
\filenq{.py} file. In fact the name of the file doesn't matter, and
the file referred to in the URL doesn't have to exist. So, given the
above configuration, \samp{http://myserver/mywebdir/myscript.py} and
\samp{http://myserver/mywebdir/montypython.py} would give the exact
same result. The important thing to understand here is that a handler
augments the server behaviour when processing a specific type of file,
not an individual file.

\emph{At this point, if you didn't understand the above paragraph, go
  back and read it again, until you do.}

\section{Now something More Complicated - Authentication\label{tut-more-complicated}}

Now that you know how to write a primitive handler, let's try
something more complicated.

Let's say we want to password-protect this directory. We want the
login to be \samp{spam}, and the password to be \samp{eggs}.

First, we need to tell Apache to call our \emph{authentication}
handler when authentication is needed. We do this by adding the
\code{PythonAuthenHandler}. So now our config looks like this:

\begin{verbatim}
  <Directory /mywebdir>
      AddHandler mod_python .py
      PythonHandler myscript
      PythonAuthenHandler myscript
      PythonDebug On
  </Directory>
\end{verbatim}

Notice that the same script is specified for two different
handlers. This is fine, because if you remember, mod_python will look
for different functions within that script for the different handlers.

Next, we need to tell Apache that we are using Basic HTTP
authentication, and only valid users are allowed (this is fairly basic
Apache stuff, so we're not going to go into details here). Our config
looks like this now:

\begin{verbatim}
  <Directory /mywebdir>
     AddHandler mod_python .py
     PythonHandler myscript
     PythonAuthenHandler myscript
     PythonDebug On
     AuthType Basic
     AuthName "Restricted Area"
     require valid-user
  </Directory>
\end{verbatim}          

Now we need to write an authentication handler function in
\file{myscript.py}. A basic authentication handler would look like
this:

\begin{verbatim}

from mod_python import apache

def authenhandler(req):

    pw = req.get_basic_auth_pw()
    user = req.user

    if user == "spam" and pw == "eggs":
       return apache.OK
    else:
       return apache.HTTP_UNAUTHORIZED
\end{verbatim}  

Let's look at this line by line: 

\begin{itemize}

\item
  \begin{verbatim}
def authenhandler(req):
  \end{verbatim}

  This is the handler function declaration. This one is called
  \code{authenhandler} because, as we already described above,
  mod_python takes the name of the directive
  (\code{PythonAuthenHandler}), drops the word \samp{Python} and converts
  it lower case.

\item
  \begin{verbatim}
    pw = req.get_basic_auth_pw()
  \end{verbatim}
  
  This is how we obtain the password. The basic HTTP authentication
  transmits the password in base64 encoded form to make it a little
  bit less obvious. This function decodes the password and returns it
  as a string. Note that we have to call this function before obtaining
  the user name.

\item
  \begin{verbatim}
    user = req.user
  \end{verbatim}
  
  This is how you obtain the username that the user entered. 

\item
  \begin{verbatim}
    if user == "spam" and pw == "eggs":
        return apache.OK
  \end{verbatim}

  We compare the values provided by the user, and if they are what we
  were expecting, we tell Apache to go ahead and proceed by returning
  \constant{apache.OK}. Apache will then consider this phase of the
  request complete, and proceed to the next phase. (Which in this case
  would be \function{handler()} if it's a \code{.py} file).

\item
  \begin{verbatim}
    else:
        return apache.HTTP_UNAUTHORIZED 
  \end{verbatim}

  Else, we tell Apache to return \constant{HTTP_UNAUTHORIZED} to the
  client, which usually causes the browser to pop a dialog box asking
  for username and password.

\end{itemize}

