\chapter{Windows Installation\label{app-wininst}}
\indexii{windows}{installation}

Notes originally created by Enrique Vaamonde \email{evaamo@loquesea.com} 

\emph{Your mileage may vary with these instructions}

You need to have the following packages properly installed and
configured in your system:

\begin{itemize}

\item
Python 1.5.2 or 2.0
\item
Apache 1.3
\item
Winzip 6.x or later. 

\end{itemize}


You need to download both the mod_python.dll and the mod_python-x.tgz
(where x is the version number) files from the main page.  Once you
have all the things above mentioned we're good to go.

\begin{enumerate}

\item
Installing mod_python libraries

\begin{itemize}

\item
Use Winzip to extract the distribution file (mod_python-x.tgz) into a
temporary folder (i.e \code{C:\e temp}):

\item
NOTE: If Winzip shows this warning "Archive contains one file, should
Winzip decompress it to a temporary folder?" just click on Yes, the
content of the file should appear in Winzip right after.

\item
Select all the files in Winzip and click on the Extract button, then
type-in the path or just browse your way to the temporary folder and
click extract.

\item       
Open your Windows Explorer and locate the temporary folder where you
extracted the distribution file, you should have a new folder in your
temporary folder (\code{C:\e temp\e mod_python-x}).

\item            
Move (or just drag \& drop) the mod_python-x folder into the Python lib
folder (i.e \code{C:\e Program Files\e Python\e lib}).

\item            
Move the files in the folder lib inside the mod_python folder
(\code{C:\e Program Files\e Python\e lib\e mod_python-x\e lib\e mod_python}) to the
\code{C:\e Program Files\e Python\e lib\e mod_python} folder. It's safe to delete
these folders we just emptied.

\end{itemize}

\item
Integrating it with Apache

      
Once the distribution file is correctly extracted and later moved into
the Python directory, it's time to modify your Apache configuration
(httpd.conf) and integrate the server with mod_python. These are a few
steps we must do first:

\begin{itemize}

\item            
Locate the file mod_python.dll that you downloaded before and move it
to Apache's modules folder (i.e \code{C:\e Program Files\e Apache Group\e Apache\e modules}).
            
\item
Go to the Apache configuration folder (i.e \code{C:\e Program Files\e Apache Group\e Apache\e conf\e }) and edit the httpd.conf file. 
            
Add the following line in the section "Dynamic Shared Object (DSO)
Support" of the httpd.conf file:

\begin{verbatim}
LoadModule python_module modules/mod_python.dll
\end{verbatim}                    

\item
Add the following lines in the section ScriptAlias and CGI of the httpd.conf: 

\begin{verbatim}                         
<Directory "<Your Document Root>/python">
   AddHandler python-program .py
   PythonHandler mptest
   PythonDebug on
</Directory>
\end{verbatim}                
    
                    
NOTE: Replace the <Your Document Root> above with the Document Root
you specified on the DocumentRoot directive in the Apache's httpd.conf
file.
                    
\item
Last, create a folder under your Document Root called python. 

\end{itemize}

\item
Testing

\begin{itemize}

\item            
Create a text file in the folder we created above and call it mptest.py
(you can use Notepad for this).
     
\item       
Insert the following lines and save the file (Make sure it gets saved
with the .py extension):

\begin{verbatim}                    
from mod_python import apache

def handler(req):
   req.content_type = "text/plain" 
   req.write("Hello World!")
   return apache.OK
\end{verbatim}                   

\item            
Make sure Apache is running (or launch it!) and then point your
browser to the URL referring to the mptest.py, you should see "Hello
World!".
     
\end{itemize}
\end{enumerate}
       
That's it, you're ready to roll!! If you don't see the "Hello World!"
message, the next section is for you.

