\chapter{Introduction\label{introduction}}

\section{Performance\label{intr-performance}}

Some very quick tests showed a very apparent performance increase: 

\begin{verbatim}
      Platform:       300Mhz Pentium MMX (Sony Vaio PCG-505TR), FreeBSD
      Program:        A script that first imported the standard library 
                      cgi module, then output a single word "Hello!".
      Measuring tool: ab (included with apache), 1000 requests.

      Standard CGI:   5 requests/s
      Cgihandler:     40 requests/s
      As a handler:   140 requests/s
\end{verbatim}         

\section{Flexibility\label{intr-flexibility}}

Apache processes requests in phases (e.g. read the request, parse
headers, check access, etc.). These phases can be implemented by
functions called handlers. Traditionally, handlers are written in C
and compiled into Apache modules. Mod_python provides a way to extend
Apache functionality by writing Apache handlers in Python. For a
detailed description of the Apache request processing process, see the
\citetitle[http://dev.apache.org/API.html]{Apache API Notes}.

For most programmers, the request and the authentication handlers
provide everything required.

To ease migration from CGI and Httpdapy, two handlers are provided
that simulate these environments allowing a user to run his scripts
under mod_python with (for the most part) no changes to the code.
 
\begin{seealso}
  \seeurl{http://dev.apache.org/}{Apache Developer Resources}
\end{seealso}

\section{History\label{intr-history}}

Mod_python originates from a project called Httpdapy. For a long time
Httpdapy was not called mod_python because Httpdapy was not meant to
be Apache-specific. Httpdapy was designed to be cross-platform and in
fact was initially written for the Netscape server.

This excerpt from the Httpdapy README file describes well the
challenges and the solution provided by embedding Python within the
HTTP server:

\begin{verbatim}

While developing my first WWW applications a few years back, I found
that using CGI for programs that need to connect to relational
databases (commercial or not) is too slow because every hit requires
loading of the interpreter executable which can be megabytes in size,
any database libraries that can themselves be pretty big, plus, the
database connection/authentication process carries a very significant
overhead because it involves things like DNS resolutions, encryption,
memory allocation, etc.. Under pressure to speed up the application, I
nearly gave up the idea of using Python for the project and started
researching other tools that claimed to specialize in www database
integration. I did not have any faith in MS's ASP; was quite
frustrated by Netscape LiveWire's slow performance and bugginess; Cold
Fusion seemed promising, but I soon learned that writing in html-like
tags makes programs as readable as assembly. Same is true for
PHP. Besides, I *really* wanted to write things in Python.

Around the same time the Internet Programming With Python book came
out and the chapter describing how to embed Python within Netscape
server immediately caught my attention.  I used the example in my
project, and developed an improved version of what I later called
Nsapy that compiled on both Windows NT and Solaris.

Although Nsapy only worked with Netscape servers, it was a very
intelligent generic OO design that, in the spirit of Python, that lent
itself for easy portability to other web servers.

Incidently, the popularity of Netscape's servers was taking a turn
south, and so I set out to port Nsapy to other servers starting with
the most popular one, Apache. And so from Nsapy was born Httpdapy.

\end{verbatim}

...continuing this saga, I later learned that writing Httpdapy for
every server is a task a little bigger and less interesting than I
originally imagined.

Instead, it seemed like providing a Python counterpart to the popular
Perl Apache extension mod_perl that would give Python users the same
(or better) capability would be a much more exciting thing to do.

And so it was done. 

